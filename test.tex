%This template is generated by Alan(Zhouqiang) Tu
%-----------------------------------------------
%	The template includes:
%	1. Sections and contents
%	2. A4 sized format for the dissertation, this part may be changed and I need to ask Fabio about it
%	3. Inserting tables
%	4. Inserting pictures. The picture should be *.eps format, and if anyone find out how to insert 
%	   a *.jpg file correctly, please do tell us.
%   5. Many LateX system do not support Asian letters(CHS, JP, etc.), so all the comments should be English, or pinyin if you like...
%-----------------------------------------------
%   UPDATES:
%	2015-03-11 09:35:043: Solve the problem of the caption for the pictures: the resolution ratio of the picture should be smaller than the size 
%                         of your screen, or the caption cannot be shown.
%-----------------------------------------------
%Last modified: 2015-03-11 09:34:34


%---------------------------------------------------------------------------------------------------------------
%						This part is used to import packages and make options to the format
%---------------------------------------------------------------------------------------------------------------
\documentclass[12pt,a4paper]{article} 								%Set the size of page to A4
\usepackage{geometry}   											%Import the geometry macro pack to manage the size of the articles
\usepackage{titlesec}  												%This macro pack is used to set the size of the header and the footer
\usepackage{graphicx}												%Enable you to use .jpg formated pictures
\setlength{\parindent}{0pt} 										%This is an option for no double-blank at the beginning of each paragraph
\geometry{left=3cm,right=2.5cm,top=1.5cm,bottom=1.5cm}			    %A normal size of the A4 paged article
\usepackage{multirow}												%Macro for multiple lines in the table
%This is the modification of the format
%---------------------------------------------------------------------------------------------------------------


%---------------------------------------------------------------------------------------------------------------
%			This part is used to set the informations, including the name of the author and other stuff
%---------------------------------------------------------------------------------------------------------------
\author{Author name} 												%The author`s name
\title{Your title}												    %The title of your article
%This is the end of this session
%---------------------------------------------------------------------------------------------------------------


%---------------------------------------------------------------------------------------------------------------
%										This is the begin of you article
%---------------------------------------------------------------------------------------------------------------
\begin{document} 													%THIS IS THE BEGIN OF YOUR ARTICLE
\maketitle 															%Introduce the subtitle system
%---------------------------------------------------------------------------------------------------------------


\section{Title One} 												%First level of title
\paragraph{Title of paragraph one}This is the content of paragraph one.
\subparagraph{Title of paragraph two} This is the content of paragraph two. \\

Lorem Ipsum is simply dummy text of the printing and typesetting industry Lorem Ipsum has been the industry's standard dummy text ever since the 1500s, when an unknown printer took a galley of type and scrambled it to make a type specimen book. It has survived not only five centuries, but also the leap into electronic typesetting, remaining essentially unchanged. It was popularised in the 1960s with the release of Letraset sheets containing Lorem Ipsum passages, and more recently with desktop publishing software like Aldus PageMaker including versions of Lorem Ipsum.\\

The most famous equation in the world: $E^2 = (m_0c^2)^2 + (pc)^2$, which is 
known as the \textbf{energy-mass-momentum} relation as an in-line equation.\\

A {\em \LaTeX{} class file}\index{\LaTeX{} class file@LaTeX class file} is a file, which holds style information for a particular \LaTeX{}.

%---------------------------------------------------------------------------------------------------------------
%								This part is used for showing how to inserting pictures
%---------------------------------------------------------------------------------------------------------------
\section{Inserting Pictures}									   %Be patient about this part
																   %This is the main reason why not many people use latex
\begin{figure}[htpb]
%\hskip -0.5in (a) 
\centering
\includegraphics[width=0.5\textwidth]{aa.eps}


\caption{This is the caption of the picture}	   %The caption cannot be seen in my computer
\label{fig:aa}
\end{figure}


\subsection{Title Two}											    %Second level of title
%---------------------------------------------------------------------------------------------------------------
%										This is the way to draw a table
%---------------------------------------------------------------------------------------------------------------
\begin{table}[!hbp]													%hbp means that the table will try to put itself on the head first
\centering															%The table will be placed in the center
\begin{tabular}{|c|c|c|c|c|}										%Start drawing the table, {|c|c|c|c|c|}indicates how many columes will be
																	%c means the elements will be in the center
\hline																%You can edit some parts of the macros to see what will be changed
lable 1-1 & label 1-2 & label 1-3 & label 1 -4 & label 1-5 \\
\hline
label 2-1 & label 2-2 & label 3-3 & label 4-4 & label 5-5 \\
\hline
\multirow{2}{*}{Multi-Row} & \multicolumn{2}{|c|}{Multi-Column} & \multicolumn{2}{|c|}{\multirow{2}{*}{Multi-Row and Col}} \\
\cline{2-3}
& column-1 & column-2 & \multicolumn{2}{|c|}{}\\
\hline
\end{tabular}
\caption{This is the title of the table}
\end{table}
%---------------------------------------------------------------------------------------------------------------


\end{document}